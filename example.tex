\documentclass{minimal}

% Transactiondiagram requires Tikz.
\usepackage{tikz}

% Tikz libraries used by transactiondiagram.
\usetikzlibrary{arrows,backgrounds,calc,decorations.pathmorphing,fit,shapes.geometric,matrix,shapes,patterns,positioning}

\usepackage{transactiondiagram}

\begin{document}

% This is an example showing the basic features.
\begin{tikzpicture}
     % Transaction, basically a line starting at point (0,0) and ending at
     % point (5,0). It contains four operations.
     \draw
           % First operation in the transaction. Each operation contains at
           % least two nodes. The first one (aop) is the label above the node.
           % It says what type of operation this is (here: start). 
           (0,0) node[aop]      {$\mathit{start}_i$}

           % The second one is the symbol that represents the operation on the
           % line. This is a white circle in the case of dot.
                 node[dot]      {} 

           % The third node is only present in the first node of each
           % transaction and contains the label that is typeset on the left of
           % the transaction. It names the transaction as Ti.
                 node[tid]      {$T_i$}

           % There is a line from the first to the second oepration, indicated
           % --. If -- was not here, there would be no line drawn between the
           % operations. The operation will appear 1 unit to the east of the
           % first operation. This operation is typical, in that it has two
           % nodes.
      -- ++(1,0) node[aop]      {$r_i(x)u_i$}
                 node[dot]      {}

           % The third operation is split into an invocation and a response.
           % This means that there will be a separate point for the invocation
           % and a separate one for the response. The invocation will be denoted
           % on the line as a half circle, so we use inv instead of dot.
           % We name the point on the line as in, by giving this label in the
           % parentheses.
      -- ++(1,0) node[aop]      {$w_i(x)v_i$}
                 node[inv] (in) {}

            % The response is one unit to the east of the invocation. We do not
            % draw a line between the invocation and the response, because we
            % will fill the gap later with a dashed line. The response is also
            % a half circle, so we use res instead of dot. We name the point on
            % the line as re.
         ++(1,0) node[aop]      {$\to\!\mathit{ok}_i$}
                 node[res] (re) {}

            % We add a note below the last operation.
      -- ++(1,0) node[aop]      {$\mathit{tryC}_i$}
                 node[dot]      {}         
                 node[not]      {this is a note}
                 ;

    % Next, we fill the gap between the invocation event at node in and the
    % response event at node re using the wait-style line.
    \draw[wait] (in) -- (re);

    % Another transaction, it starts at point (4,-1), so below and to the east
    % of where the previous transaction started.
    \draw
          (4,-1) node[aop]      {$\mathit{start}_j$}
                 node[dot]      {} 
                 node[tid]      {$T_j$}

      -- ++(1,0) node[aop]      {$r_j(x)$}
                 node[inv] (in2){}

         ++(1,0) node[aop]      {$\to\!v_j$}
                 node[res] (re2){}

      -- ++(1.5,0) node[aop]    {$\mathit{tryC}_j$}
                 node[dot] (ab) {}       
                 ;

    % A dashed line between two events of a split operation
    \draw[wait] (in2) -- (re2);

    % We draw a happens-before arrow between the invocation events of the two
    % transactions. The line is of class hb and is defined by the command
    % \squiggle.
    \draw[hb] (in) \squiggle (in2);

    % Next we draw another transaction in the same line as the previous one
    \draw
          (9,-1) node[aop]      {$\mathit{start}_j'$}
                 node[dot] (st) {} 
                 node[tid]      {$T_j'$}


      -- ++(1,0) node[aop]      {$r_j'(x)v_k$}
                 node[dot] (r)  {}

      % In this transaction we add a note below the last operation. We use
      % mutlinot to split the text into several lines.
      -- ++(1,0) node[aop]      {$\mathit{tryC}_j'$}
                 node[dot] (n1) {}         
                 node[not]      {\multinot{this is a \\ 2-line note}}
                 ;

    % We connect the two transactions by a snaky line from the last operation
    % of the first transaction to the first operation of the second
    % transaction. The line has class retry.
    \draw[retry] (ab) -- (st);

    \draw
          (2,-2) node[aop]      {$\mathit{start}_j$}
                 node[dot]      {} 
                 node[tid]      {$T_k$}

      -- ++(1,0) node[aop]      {$w_k(x)v_k$}
                 node[inv] (in3){}

         ++(3.25,0) node[aop]   {$\to\!\mathit{ok}_k$}
                 node[res] (re3){}

         % In the third transaction we want the text to be below the point on
         % the line, so we change aop to bop.
      -- ++(1,0) node[bop]      {$\mathit{tryC}_k$}
                 node[dot] (co) {}         
                 ;

    \draw[wait] (in3) -- (re3);
   
    % We draw another happens before relation, but this one is inclining rather
    % then declining, so we use squiggleup instead of squiggle.
    \draw[hb] (co) \squiggleup (r);

    % Axis with numbers on it.
    \axis{0}{12}{-3}

    % Alternatively, just a simple axis with just an arrow:
    % \simpleaxis{0}{12}{-3}
\end{tikzpicture}

\vspace{2cm}

% This is an example of using rectangles to indicate stretches of time. It
% shows two transactions, where one is longer than the other, and the rectangle
% is drawn over the longer of the two to emphasise the difference.
\begin{tikzpicture}
     % The transaction is pretty ordinary, but we add an unmarked node (without
     % any style) and name nodes b0, b1, and b2. The umarked node will give a
     % rectangle a place to "attach".
     \draw
           (0,0) node[bop]      {$\mathit{start}_i$}
                 node[dot] (b0) {} 
                 node[tid]      {$T_i$}

      -- ++(2,0) node      (b1) {}

      -- ++(2,0) node[aop]      {$r_i(x)u_i$}
                 node[dot]      {}

      -- ++(2,0) node[aop]      {$w_i(x)v_i$}
                 node[dot]      {}

      % The unmarked node.

      -- ++(2,0) node[aop]      {$\mathit{tryC}_i$}
                 node[dot] (b2) {}         
                 ;

     % The first rectangle we mark is going to show the width of the difference
     % between the two transactions. Hence we put down a rectangle starting at
     % node b0 and ending at b1.
     \getrect{b0}{b1}
             % The node will be 1cm high and go up. If it were going down it'd
             % be defined as {-1cm}{below}.
             {1cm}{above}
             % The rectangle is defined to be hatched using gray perpendicular
             % lines, as per standard tikz magic.
             {black, pattern=north west lines, pattern color=gray}
             % At the top of the rectangle there are two text labels, one at
             % the beginning saying $\delta_1$, and one at the end saying
             % $\delta_2$.
             {$\delta_1$}{$\delta_2$};

     % We also draw a smaller rectangle at the end. Since there is no
     % difference at the end, we indicate that by drawing a slim rectangle over
     % a sinlge node, b2. 
     \getslimrect{b2}
             {1cm}{above}
             {black, pattern=north west lines, pattern color=gray}
             {$\delta_3$};

    \draw
           (2,-1) node[aop]      {$\mathit{start}_i$}
                 node[dot]      {} 
                 node[tid]      {$T_i'$}

      -- ++(2,0) node[aop]      {$r_i(x)u_i$}
                 node[dot]      {}

      -- ++(2,0) node[aop]      {$w_i(x)v_i$}
                 node[dot]      {}

      -- ++(2,0) node[aop]      {$\mathit{tryC}_i$}
                 node[dot]      {}         
                 ;



    \axis{0}{9}{-2}

    % Alternatively, just a simple axis with just an arrow:
    % \simpleaxis{0}{12}{-3}
\end{tikzpicture}


\end{document}
